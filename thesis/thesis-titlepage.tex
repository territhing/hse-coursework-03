\begin{titlepage}
\newpage

{\setstretch{1.0}
\begin{center}
Федеральное государственное автономное образовательное учреждение высшего образования «Национальный исследовательский университет «Высшая школа экономики»
\\
\bigskip
Факультет компьютерных наук \\
Основная образовательная программа \\
Прикладная математика и информатика \\
\end{center}
}

\vspace{8em}

\begin{center}
{\Large КУРСОВАЯ РАБОТА}\\
\textsc{\textbf{
Исследовательский проект на тему
\linebreak
\enquote{Модели глубинного обучения для автоматической расстановки знаков пунктуаций в последовательности слов}}}
\end{center}

\vspace{2em}

{\setstretch{1.0}
\hfill\parbox{16cm}{
\hspace*{5cm}\hspace*{-5cm}Выполнил студент группы 191, 3 курса,\\
 Пилипенко Сергей Владимирович\\
 
\hspace*{5cm}\hspace*{-5cm}Руководитель КР:\\
стажер-исследователь Яруллин Рамиль Ильдарович\\

%\hspace*{5cm}\hspace*{-5cm}Куратор:\hfill < степень>, <звание>, <ФИО полностью>\\

% \hspace*{5cm}\hspace*{-5cm}Консультант:\\
% научный сотрудник Лобачева Екатерина Максимовна\\
}
}

\vspace{\fill}

\begin{center}
Москва 2021
\end{center}

\end{titlepage}